

\subsection{Signatures}%
\label{sec:background}



The concept of the signature approach is to extract characteristic features from a data, that is a function or data points in a non parametric way.
First we want to define the so-called path integral.
Consider a two parameterized one dimensional paths  $Y_{t}: [a,b] \to  \mathbb{R} ^{d} $ and $X_t: [a,b] \to \mathbb{R} $, then we say the path integral of $Y_{t}$ against $X_{t}$  is \begin{equation}
    \int_{a}^{b} Y_{t} dX_{t} = \int_{a}^{b}  Y_{t} \dot{X_{t}}dt,
\end{equation}  where we defined $ \dot{X_{t}} = \frac{d}{dt}X( t)   $.



A fundamental piece in the definition of a signature is
the so-called path integral.
Lets consider the parametrized smooth path of $d$ dimensions be $X_{t}: \left[ a,b \right] \to  \mathbb{R} ^{d}  $ such that $X_{t} = \left\{ X^{1}_{t},  X_{t}^{2}, \ldots, X_{t}^{d} \right\} $. Now since each path is $X^{i}_{t}: [a,b] \to
\mathbb{R} $ for $i \in \left\{ 1, \ldots, d \right\} $ , we say define the integral \begin{equation}
    S( X)_{a,t}^{i} =  \int_{a < s < t}^{} dX^{i} = X^{i}_{t} - X_{0}^{i}
\end{equation}

Similarly we define the double-iterated double integral \begin{equation*}
    S( X)^{i,j}_{a,t} = \int_{a < s<t}^{} S( X)_{a,s}^{i}dX^{j}_{s} = \int_{\substack{a < r < t \\ a < s < t}}^{} dX^{i}_{r} dX^{j}_{s}
\end{equation*}


 Continuing recursively we obtain the definition \begin{equation*}
     S( X) _{a,t}^{i_{1},\ldots, i_{k}} =  \int_{a < s <t}^{} S( X) ^{i_{1}, \ldots, i_{k-1}} dX^{i_{k}}_{s}
 \end{equation*}
 where $i_{1}, \ldots, i_{k} \in \left\{ 1, \ldots, d \right\} $. Notice that we still obtain the mapping $S( X)^{i_{1}, \ldots, i_{k}}: [a,b] \to \mathbb{R}  $.  Finally we have the tools rquired to define a signature.

 \begin{definition}[Signature]
     We say a signature of a path $X: \left[ a,b \right] \to  \mathbb{R} ^{d} $, denoted by $S( X) _{a,b}$ is the collection of all the iterated integrals of $X$. Thus we, nor have the sequence of numbers \begin{equation}
     \begin{split}
         S( X) _{a,b}  = & ( 1,  S( X) ^{1}_{a,b}, \ldots, S( X) ^{d}_{a,b}, S( X) ^{1,1}_{a,b}, \\ & \ \  S( X) ^{1,2} _{a,b}, S( X) _{a,b}^{2,1}, \ldots).
     \end{split}
     \end{equation}

     Here the first term is defined as $1$. Keep in mind that we iterate over all multi-indexes, that is the set \begin{equation}
W =\left\{
         \begin{split}
           ( i_{1}, \ldots, i_{k})  \text{ where }  k \ge 1, \\
         \text{ for all } i_{1}, \ldots, i_{k} \in \left\{ 1,\ldots,d \right\} .
         \end{split}
\right\}
     \end{equation}

     We denote the set $W$ as words and $ A = \left\{ 1, \ldots, d \right\} $ as the alphabet of $d $ letters .
 \end{definition}


One of the most fundamental properties of the signature is its invariance under time reparameterization. This can easily be demonstrated using the definitions of the path integral. Consider two paths $X,Y: \left[ a,b \right] \to \mathbb{R}$, which
are real-valued. Now, consider two corresponding reparameterized paths $\widetilde{X}, \widetilde{Y}: \left[ a,b \right] \to \mathbb{R}$, where $\widetilde{X}_{t} = X_{\psi(t)}$ and $\widetilde{Y} = Y_{\psi(t)}$, with some smooth reparameterization $\psi: [a,b]
\to [a,b]$. From the chain rule it is clear that $$\frac{d}{dt} \widetilde{X}_{ t } = \dot{ \widetilde{X}_{ t }}  \dot{ \psi} ( t)  $$, thus it follows that \begin{equation}
    \int_{a}^{b }  \widetilde{Y}_{t} d\widetilde{X}_{t} = \int_{a}^{b}  Y_{\psi ( t) }\dot{X} _{\psi ( t) } \dot{\psi}( t) = \int_{a}^{b}  Y_{u} dX_{u}
\end{equation}

Here we used the subsitution $u = \psi ( t)  $. This is of course applicable in the multidimensional case in the case of a signature. Let $\widetilde{X},X: \left[ a,b \right] \to \mathbb{R} d$ where $\widetilde{X}_{t} = X_{t}$.Then we see that \begin{equation}
    S( \widetilde{X}) _{a,b}^{i_{1},\ldots, i_{k}} = S( X) ^{i_{1},\ldots, i_{k}}_{a,b}
\end{equation}

for any $i_{1},\ldots, i_{k} \in  \left\{ 1,\ldots,d \right\} $. Thus we see that the signature is, in fact, invariant under time re parameterization.

\subsection{Shuffle product}%
\label{sub:shuffle_product}


A fundamental property of the signature is that the product of two signature terms \( S(X)_a^{i_1,\dots,i_k,b} \) and \( S(X)_a^{j_1,\dots,j_m,b} \) can be expressed as a sum of terms depending on shuffled multi-indexes.

To formalize this, we define the shuffle product for two multi-indexes. A permutation \( \sigma \) of the set \( \{1, \dots, k+m\} \) is called a \((k,m)\)-shuffle if \( \sigma^{-1}(1) < \dots < \sigma^{-1}(k) \) and \( \sigma^{-1}(k+1) < \dots < \sigma^{-1}(k+m) \). The list \( (\sigma(1), \dots, \sigma(k+m)) \) forms a shuffle of \( (1, \dots, k) \) and \( (k+1, \dots, k+m) \). Let \( \text{Shuffles}(k,m) \) denote the collection of all such shuffles.

For multi-indexes \( I = (i_1, \dots, i_k) \) and \( J = (j_1, \dots, j_m) \) with \( i_1, \dots, i_k, j_1, \dots, j_m \in \{1, \dots, d\} \), define the multi-index
\[
 I \shuffle J =\left\{   (\sigma(1), \dots, \sigma(k+m)) \mid \sigma \in \text{Shuffles}(k,m) \right\} .
\]
Thus, we have that The shuffle product \( I \shuffle J \) is the set of multi-indexes of length \( k+m \).

\begin{theorem}
\textbf{Shuffle product identity} For a path \( X: [a,b] \to \mathbb{R}^d \) and multi-indexes \( I = (i_1, \dots, i_k) \) and \( J = (j_1, \dots, j_m) \), it holds that
\[
S(X)_a^b S(X)_a^b = \sum_{K \in I \shuffle J} S(X)_a^{K,b}.
\]
\end{theorem}




Let the terms $e_{i_{1}}, \ldots, e_{i_{k}} $ be monomials. Then we can denote the representation $S( X) _{a,b}$  as a formal power series \begin{equation}
    \label{eq:representation}
S(X )  _{a,b} = \sum_{k=0}^{\infty}  \sum_{i_{1},\ldots , i_{k}}^{} S( X) _{a,b}^{i_{1},\ldots, i_{k}} e_{i_{1}} \cdot \ldots\cdot e_{i_{k}}
\end{equation}

The main reason why this is fundamental is because this is necessary to state the so-called Chens identity which states the relationship between concatenation and tensor product. First we define the concatenation of paths. That is. For two paths $X: \left[ a,b \right] \to \mathbb{R} ^{d}$  and $Y : \left[ b,c \right]
\to  \mathbb{R} ^{d}$, we define the concatenation as the path $X * Y : \left[ a,c \right]  \to \mathbb{R} ^{d}$, where the first part for $t \in  \left[ a,b \right] $ is $( X* Y) _{t} = X_{t}$, and for $t \in  \left[ b,c \right] $ we define $( X*Y)
_{t} = X_{b} + ( Y_{t} - Y_{b}) $. And the tensor product is simply defined joining the monomials, that is  $e_{i_{1}} \ldots e_{i_{k}} \otimes e_{j_{1}} \ldots e_{j_{m}}  = e_{i_{1}} \ldots e_{i_{k}}  e_{j_{1}} \ldots e_{j_{m}}$. Finally we can
state the relationship between these operators.

\begin{theorem}
    \textbf{Chens Identity.} Let $X: \left[ a,b \right] \to  \mathbb{R} ^{d}$ and $Y: \left[ b,c \right] \to \mathbb{R} ^{d}$ be two paths. Then we have the following identity,
    \begin{equation}
        S( X*Y) _{a,c} = S( X) _{a,b} \otimes  S( Y) _{b,c}
    \end{equation}

\end{theorem}


A very interesting property with the signature \eqref{eq:representation} is in fact time-reversible. For a path $X: \left[ a,b \right]  \to \mathbb{R} ^{d}$ we can define the time-reverse path $\overleftarrow{X}: \left[ a,b \right]  \to  \mathbb{R} ^{d} $, for
which $ \overleftarrow{X}  = X_{a+b -t}$. It can be shown that \begin{equation}
    S( X) _{a,b} \otimes S( \overleftarrow{X}) = 1.
\end{equation}
This can be understood that that the time-reverse is the "tensor" inverse for the signature $S( X) _{a,b}$.



