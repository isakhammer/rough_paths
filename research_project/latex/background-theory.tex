

\section{Background Theory}%
\label{sec:background}


The concept of the signature approach is to extract characteristic features from a data, that is a function or data points in a non parametric way.
First we want to define the so-called path integral.
Consider a two parameterized one dimensional paths  $Y_{t}: [a,b] \to  \mathbb{R} ^{d} $ and $X_t: [a,b] \to \mathbb{R} $, then we say the path integral of $Y_{t}$ against $X_{t}$  is \begin{equation}
    \int_{a}^{b} Y_{t} dX_{t} = \int_{a}^{b}  Y_{t} \dot{X_{t}}dt,
\end{equation}  where we defined $ \dot{X_{t}} = \frac{d}{dt}X( t)   $.


A fundamental piece in the definition of a signature is
the so-called path integral.
Lets consider the parametrized smooth path of $d$ dimensions be $X_{t}: \left[ a,b \right] \to  \mathbb{R} ^{d}  $ such that $X_{t} = \left\{ X^{1}_{t},  X_{t}^{2}, \ldots, X_{t}^{d} \right\} $. Now since each path is $X^{i}_{t}: [a,b] \to
\mathbb{R} $ for $i \in \left\{ 1, \ldots, d \right\} $ , we say define the integral \begin{equation}
    S( X)_{a,t}^{i} =  \int_{a < s < t}^{} dX^{i} = X^{i}_{t} - X_{0}^{i}
\end{equation}

Similarly we define the double-iterated double integral \begin{equation*}
    S( X)^{i,j}_{a,t} = \int_{a < s<t}^{} S( X)_{a,s}^{i}dX^{j}_{s} = \int_{\substack{a < r < t \\ a < s < t}}^{} dX^{i}_{r} dX^{j}_{s}
\end{equation*}


 Continuing recursively we obtain the definition \begin{equation*}
     S( X) _{a,t}^{i_{1},\ldots, i_{k}} =  \int_{a < s <t}^{} S( X) ^{i_{1}, \ldots, i_{k-1}} dX^{i_{k}}_{s}
 \end{equation*}
 where $i_{1}, \ldots, i_{k} \in \left\{ 1, \ldots, d \right\} $. Notice that we still obtain the mapping $S( X)^{i_{1}, \ldots, i_{k}}: [a,b] \to \mathbb{R}  $.  Finally we have the tools rquired to define a signature.

 \begin{definition}[Signature]
     We say a signature of a path $X: \left[ a,b \right] \to  \mathbb{R} ^{d} $, denoted by $S( X) _{a,b}$ is the collection of all the iterated integrals of $X$. Thus we, nor have the sequence of numbers \begin{equation}
     \begin{split}
         S( X) _{a,b}  = & ( 1,  S( X) ^{1}_{a,b}, \ldots, S( X) ^{d}_{a,b}, S( X) ^{1,1}_{a,b}, \\ & \ \  S( X) ^{1,2} _{a,b}, S( X) _{a,b}^{2,1}, \ldots).
     \end{split}
     \end{equation}

     Here the first term is defined as $1$. Keep in mind that we iterate over all multi-indexes, that is the set \begin{equation}
W =\left\{
         \begin{split}
           ( i_{1}, \ldots, i_{k})  \text{ where }  k \ge 1, \\
         \text{ for all } i_{1}, \ldots, i_{k} \in \left\{ 1,\ldots,d \right\} .
         \end{split}
\right\}
     \end{equation}

     We denote the set $W$ as words and $ A = \left\{ 1, \ldots, d \right\} $ as the alphabet of $d $ letters .
 \end{definition}


 One of the most fundemental properties of the signature is the invariance of time reparameterization. This can easily be demonstrated using the definitions of the path integral. Consider two paths $X,Y: \left[ a,b \right]  \to \mathbb{R} $ be two
 real-valued paths and $ $
