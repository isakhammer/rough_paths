

\section{Background Theory}%
\label{sec:background}


The concept of the signature approach is to extract characteristic features from a data, that is a function or data points in a non parametric way.
First we want to define the so-called path integral.
Consider a two parameterized one dimensional paths  $Y_{t}: [a,b] \to  \mathbb{R} ^{d} $ and $X_t: [a,b] \to \mathbb{R} $, then we say the path integral of $Y_{t}$ against $X_{t}$  is \begin{equation}
    \int_{a}^{b} Y_{t} dX_{t} = \int_{a}^{b}  Y_{t} \dot{X_{t}}dt,
\end{equation}  where we defined $ \dot{X_{t}} = \frac{d}{dt}X( t)   $.


A fundamental piece in the definition of a signature is
the so-called path integral.
Lets consider the parametrized smooth path of $d$ dimensions be $X_{t}: \left[ a,b \right] \to  \mathbb{R} ^{d}  $ such that $X_{t} = \left\{ X^{1}_{t},  X_{t}^{2}, \ldots, X_{t}^{d} \right\} $. Now since each path is $X^{i}_{t}: [a,b] \to
\mathbb{R} $ for $i \in \left\{ 1, \ldots, d \right\} $ , we say define the integral \begin{equation}
    S( X)_{a,t}^{i} =  \int_{a < s < t}^{} dX^{i} = X^{i}_{t} - X_{0}^{i}
\end{equation}



