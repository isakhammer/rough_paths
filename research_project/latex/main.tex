

% \documentclass[12pt]{article}


% https://cloudnet2016.ieee-cloudnet.org/authors/final-submission/
\documentclass[10pt,conference]{IEEEtran}


\usepackage{isaks_template} % Contains all included packages. See isaks_template.sty.
\usepackage{stix}

% latex margins
% \linespread{1.5}
\newgeometry{vmargin={15mm}, hmargin={25mm,37mm}}
%
\title{ {\Large \textbf{Signatures and its application to rough path theory  }} }
% IF ONE AUTHOR
%\address{Norwegian University of Science and Technology \\
%Department of Mathematical Sciences \\
%{\tt isakhammer@gmail.com}}
%

\begin{document}
\author{
Isak Hammer$^{\star\dagger}$  \\
% {\small Supervisor: André Massing$^{\dagger} $  } \\
{\small  August, 2024}\\
{\footnotesize $^\star$isakhammer@gmail.com }
% \vspace{7mm} %5mm vertical space
{\footnotesize $^\dagger$Department of Mathematics, UiB}\\
}

\maketitle
% Comment this out to remove todos
% https://tex.stackexchange.com/questions/4830/how-to-hide-todo-notes-without-deleting-them-manually
% \renewcommand{\todo}[1]{}
\begin{sloppy}
 \textit{ \textbf{Note.} This article is submitted as an examination in the course Algebraic structures for differential equations, computations and flows for the spring semester 2024 at the Department of Mathematics, UiB \\}

\begin{abstract} Signatures have emerged as a powerful tool in machine learning, particularly for efficiently capturing key geometric features of data. In this report, we introduce the foundational concepts of signatures, outlining their essential
properties such as invariance under time reparametrization, time-reversibility, and others. We further explore various applications of signatures, highlighting their versatility and utility. Finally, we revise the signature uniqueness theorem, which underscores the theoretical significance of signatures in the context of rough path theory.
\end{abstract}

    \section{Introduction}\label{sec:introduction}


Signatures has proved to be fundamental for several applications.
The theory is quite applicable in the algebraic theory of rough paths \cite{geng2021introduction}. This has showed to be a very interesting way to represent data because of the special properties of time reversial propertie and ivariance of time
reparamerization of representing data via signatures \cite{chevyrev2016primer}. In fact, in a
2013 competition focused on recognizing handwritten
Chinese characters, the winner represented the characters as arrays based on a "signature" from rough path theory, then classified them using a convolutional neural network \cite{yin2013icdar}.
Recently have we sen that it is clear that signatures can be a great tool of extracting geometric shape of any path given missing data.

Recently, it has become clear that signatures can be a powerful representation in machine learning for extracting the geometric shape of any path, even when data is missing \cite{chevyrev2016primer} and time series.


We will divide the report into two parts. One part for the fundamental properties of signatures and the second pat for the recent applications of this method. Spend some time evaluating signatures for


\todo[inline]{ Main idea is that the signature approach represent a non-parametric way of extraction of characteristics features of data.  }



    

\subsection{Signatures}%
\label{sec:background}



The concept of the signature approach is to extract characteristic features from a data, that is a function or data points in a non parametric way.
First we want to define the so-called path integral.
Consider a two parameterized one dimensional paths  $Y_{t}: [a,b] \to  \mathbb{R} ^{d} $ and $X_t: [a,b] \to \mathbb{R} $, then we say the path integral of $Y_{t}$ against $X_{t}$  is \begin{equation}
    \int_{a}^{b} Y_{t} dX_{t} = \int_{a}^{b}  Y_{t} \dot{X_{t}}dt,
\end{equation}  where we defined $ \dot{X_{t}} = \frac{d}{dt}X( t)   $.



A fundamental piece in the definition of a signature is
the so-called path integral.
Lets consider the parametrized smooth path of $d$ dimensions be $X_{t}: \left[ a,b \right] \to  \mathbb{R} ^{d}  $ such that $X_{t} = \left\{ X^{1}_{t},  X_{t}^{2}, \ldots, X_{t}^{d} \right\} $. Now since each path is $X^{i}_{t}: [a,b] \to
\mathbb{R} $ for $i \in \left\{ 1, \ldots, d \right\} $ , we say define the integral \begin{equation}
    S( X)_{a,t}^{i} =  \int_{a < s < t}^{} dX^{i} = X^{i}_{t} - X_{0}^{i}
\end{equation}

Similarly we define the double-iterated double integral \begin{equation*}
    S( X)^{i,j}_{a,t} = \int_{a < s<t}^{} S( X)_{a,s}^{i}dX^{j}_{s} = \int_{\substack{a < r < t \\ a < s < t}}^{} dX^{i}_{r} dX^{j}_{s}
\end{equation*}


 Continuing recursively we obtain the definition \begin{equation*}
     S( X) _{a,t}^{i_{1},\ldots, i_{k}} =  \int_{a < s <t}^{} S( X) ^{i_{1}, \ldots, i_{k-1}} dX^{i_{k}}_{s}
 \end{equation*}
 where $i_{1}, \ldots, i_{k} \in \left\{ 1, \ldots, d \right\} $. Notice that we still obtain the mapping $S( X)^{i_{1}, \ldots, i_{k}}: [a,b] \to \mathbb{R}  $.  Finally we have the tools rquired to define a signature.

 \begin{definition}[Signature]
     We say a signature of a path $X: \left[ a,b \right] \to  \mathbb{R} ^{d} $, denoted by $S( X) _{a,b}$ is the collection of all the iterated integrals of $X$. Thus we, nor have the sequence of numbers \begin{equation}
     \begin{split}
         S( X) _{a,b}  = & ( 1,  S( X) ^{1}_{a,b}, \ldots, S( X) ^{d}_{a,b}, S( X) ^{1,1}_{a,b}, \\ & \ \  S( X) ^{1,2} _{a,b}, S( X) _{a,b}^{2,1}, \ldots).
     \end{split}
     \end{equation}

     Here the first term is defined as $1$. Keep in mind that we iterate over all multi-indexes, that is the set \begin{equation}
W =\left\{
         \begin{split}
           ( i_{1}, \ldots, i_{k})  \text{ where }  k \ge 1, \\
         \text{ for all } i_{1}, \ldots, i_{k} \in \left\{ 1,\ldots,d \right\} .
         \end{split}
\right\}
     \end{equation}

     We denote the set $W$ as words and $ A = \left\{ 1, \ldots, d \right\} $ as the alphabet of $d $ letters .
 \end{definition}


One of the most fundamental properties of the signature is its invariance under time reparameterization. This can easily be demonstrated using the definitions of the path integral. Consider two paths $X,Y: \left[ a,b \right] \to \mathbb{R}$, which
are real-valued. Now, consider two corresponding reparameterized paths $\widetilde{X}, \widetilde{Y}: \left[ a,b \right] \to \mathbb{R}$, where $\widetilde{X}_{t} = X_{\psi(t)}$ and $\widetilde{Y} = Y_{\psi(t)}$, with some smooth reparameterization $\psi: [a,b]
\to [a,b]$. From the chain rule it is clear that $$\frac{d}{dt} \widetilde{X}_{ t } = \dot{ \widetilde{X}_{ t }}  \dot{ \psi} ( t)  $$, thus it follows that \begin{equation}
    \int_{a}^{b }  \widetilde{Y}_{t} d\widetilde{X}_{t} = \int_{a}^{b}  Y_{\psi ( t) }\dot{X} _{\psi ( t) } \dot{\psi}( t) = \int_{a}^{b}  Y_{u} dX_{u}
\end{equation}

Here we used the subsitution $u = \psi ( t)  $. This is of course applicable in the multidimensional case in the case of a signature. Let $\widetilde{X},X: \left[ a,b \right] \to \mathbb{R} d$ where $\widetilde{X}_{t} = X_{t}$.Then we see that \begin{equation}
    S( \widetilde{X}) _{a,b}^{i_{1},\ldots, i_{k}} = S( X) ^{i_{1},\ldots, i_{k}}_{a,b}
\end{equation}

for any $i_{1},\ldots, i_{k} \in  \left\{ 1,\ldots,d \right\} $. Thus we see that the signature is, in fact, invariant under time re parameterization.

\subsection{Shuffle product}%
\label{sub:shuffle_product}


A fundamental property of the signature is that the product of two signature terms \( S(X)_a^{i_1,\dots,i_k,b} \) and \( S(X)_a^{j_1,\dots,j_m,b} \) can be expressed as a sum of terms depending on shuffled multi-indexes.

To formalize this, we define the shuffle product for two multi-indexes. A permutation \( \sigma \) of the set \( \{1, \dots, k+m\} \) is called a \((k,m)\)-shuffle if \( \sigma^{-1}(1) < \dots < \sigma^{-1}(k) \) and \( \sigma^{-1}(k+1) < \dots < \sigma^{-1}(k+m) \). The list \( (\sigma(1), \dots, \sigma(k+m)) \) forms a shuffle of \( (1, \dots, k) \) and \( (k+1, \dots, k+m) \). Let \( \text{Shuffles}(k,m) \) denote the collection of all such shuffles.

For multi-indexes \( I = (i_1, \dots, i_k) \) and \( J = (j_1, \dots, j_m) \) with \( i_1, \dots, i_k, j_1, \dots, j_m \in \{1, \dots, d\} \), define the multi-index
\[
 I \shuffle J =\left\{   (\sigma(1), \dots, \sigma(k+m)) \mid \sigma \in \text{Shuffles}(k,m) \right\} .
\]
Thus, we have that The shuffle product \( I \shuffle J \) is the set of multi-indexes of length \( k+m \).

\begin{theorem}
\textbf{Shuffle product identity} For a path \( X: [a,b] \to \mathbb{R}^d \) and multi-indexes \( I = (i_1, \dots, i_k) \) and \( J = (j_1, \dots, j_m) \), it holds that
\[
S(X)_a^b S(X)_a^b = \sum_{K \in I \shuffle J} S(X)_a^{K,b}.
\]
\end{theorem}




Let the terms $e_{i_{1}}, \ldots, e_{i_{k}} $ be monomials. Then we can denote the representation $S( X) _{a,b}$  as a formal power series \begin{equation}
    \label{eq:representation}
S(X )  _{a,b} = \sum_{k=0}^{\infty}  \sum_{i_{1},\ldots , i_{k}}^{} S( X) _{a,b}^{i_{1},\ldots, i_{k}} e_{i_{1}} \cdot \ldots\cdot e_{i_{k}}
\end{equation}

The main reason why this is fundamental is because this is necessary to state the so-called Chens identity which states the relationship between concatenation and tensor product. First we define the concatenation of paths. That is. For two paths $X: \left[ a,b \right] \to \mathbb{R} ^{d}$  and $Y : \left[ b,c \right]
\to  \mathbb{R} ^{d}$, we define the concatenation as the path $X * Y : \left[ a,c \right]  \to \mathbb{R} ^{d}$, where the first part for $t \in  \left[ a,b \right] $ is $( X* Y) _{t} = X_{t}$, and for $t \in  \left[ b,c \right] $ we define $( X*Y)
_{t} = X_{b} + ( Y_{t} - Y_{b}) $. And the tensor product is simply defined joining the monomials, that is  $e_{i_{1}} \ldots e_{i_{k}} \otimes e_{j_{1}} \ldots e_{j_{m}}  = e_{i_{1}} \ldots e_{i_{k}}  e_{j_{1}} \ldots e_{j_{m}}$. Finally we can
state the relationship between these operators.

\begin{theorem}
    \textbf{Chens Identity.} Let $X: \left[ a,b \right] \to  \mathbb{R} ^{d}$ and $Y: \left[ b,c \right] \to \mathbb{R} ^{d}$ be two paths. Then we have the following identity,
    \begin{equation}
        S( X*Y) _{a,c} = S( X) _{a,b} \otimes  S( Y) _{b,c}
    \end{equation}

\end{theorem}


A very interesting property with the signature \eqref{eq:representation} is in fact time-reversible. For a path $X: \left[ a,b \right]  \to \mathbb{R} ^{d}$ we can define the time-reverse path $\overleftarrow{X}: \left[ a,b \right]  \to  \mathbb{R} ^{d} $, for
which $ \overleftarrow{X}  = X_{a+b -t}$. It can be shown that \begin{equation}
    S( X) _{a,b} \otimes S( \overleftarrow{X}) = 1.
\end{equation}
This can be understood that that the time-reverse is the "tensor" inverse for the signature $S( X) _{a,b}$.




    % 

% \section{Rough paths}%
% \label{sec:evolutionary_pde_s_of_the_willmore_flow}


















    \printbibliography
\end{sloppy}

\end{document}
