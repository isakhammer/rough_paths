

% \documentclass[12pt]{article}


% https://cloudnet2016.ieee-cloudnet.org/authors/final-submission/
\documentclass[10pt,conference]{IEEEtran}


\usepackage{isaks_template} % Contains all included packages. See isaks_template.sty.
\usepackage{stix}

% latex margins
% \linespread{1.5}
\newgeometry{vmargin={15mm}, hmargin={25mm,37mm}}
%
\title{ {\Large \textbf{Signatures and its application to rough path theory  }} }
% IF ONE AUTHOR
%\address{Norwegian University of Science and Technology \\
%Department of Mathematical Sciences \\
%{\tt isakhammer@gmail.com}}
%

\begin{document}
\author{
Isak Hammer$^{\star\dagger}$  \\
% {\small Supervisor: André Massing$^{\dagger} $  } \\
{\small  August, 2024}\\
{\footnotesize $^\star$isakhammer@gmail.com }
% \vspace{7mm} %5mm vertical space
{\footnotesize $^\dagger$Department of Mathematics, UiB}\\
}

\maketitle
% Comment this out to remove todos
% https://tex.stackexchange.com/questions/4830/how-to-hide-todo-notes-without-deleting-them-manually
% \renewcommand{\todo}[1]{}
\begin{sloppy}
 \textit{ \textbf{Note.} This article is submitted as an examination in the course Algebraic structures for differential equations, computations and flows for the spring semester 2024 at the Department of Mathematics, UiB \\}

\begin{abstract} Signatures have emerged as a powerful tool in machine learning, particularly for efficiently capturing key geometric features of data. In this report, we introduce the foundational concepts of signatures, outlining their essential
properties such as invariance under time reparametrization, time-reversibility, and others. We further explore various applications of signatures, highlighting their versatility and utility. Finally, we revise the signature uniqueness theorem, which underscores the theoretical significance of signatures in the context of rough path theory.
\end{abstract}

    \section{Introduction}\label{sec:introduction}

\subsection{Context}%
\label{sub:context}




\subsection{Outline of this report}%
\label{sub:outline_of_this_report}

We will divide the report into two parts. One part for the fundamental properties of signatures and the second pat for the recent applications of this method.

\todo[inline]{ Main idea is that the signature approach represent a non-parametric way of extraction of characteristics features of data.  }



    

\section{Background Theory}%
\label{sec:background}


The concept of the signature approach is to extract characteristic features from a data, that is a function or data points in a non parametric way.
First we want to define the so-called path integral.
Consider a two parameterized one dimensional paths  $Y_{t}: [a,b] \to  \mathbb{R} ^{d} $ and $X_t: [a,b] \to \mathbb{R} $, then we say the path integral of $Y_{t}$ against $X_{t}$  is \begin{equation}
    \int_{a}^{b} Y_{t} dX_{t} = \int_{a}^{b}  Y_{t} \dot{X_{t}}dt,
\end{equation}  where we defined $ \dot{X_{t}} = \frac{d}{dt}X( t)   $.


A fundamental piece in the definition of a signature is
the so-called path integral.
Lets consider the parametrized smooth path of $d$ dimensions be $X_{t}: \left[ a,b \right] \to  \mathbb{R} ^{d}  $ such that $X_{t} = \left\{ X^{1}_{t},  X_{t}^{2}, \ldots, X_{t}^{d} \right\} $. Now since each path is $X^{i}_{t}: [a,b] \to
\mathbb{R} $ for $i \in \left\{ 1, \ldots, d \right\} $ , we say define the integral \begin{equation}
    S( X)_{a,t}^{i} =  \int_{a < s < t}^{} dX^{i} = X^{i}_{t} - X_{0}^{i}
\end{equation}




    % 

\section{Rough paths}%
\label{sec:evolutionary_pde_s_of_the_willmore_flow}


















    \printbibliography
\end{sloppy}

\end{document}
