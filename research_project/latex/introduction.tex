\section{Introduction}\label{sec:introduction}

\subsection{Context}%
\label{sub:context}




\subsection{Outline of this report}%
\label{sub:outline_of_this_report}

Signatures has proved to be fundamental for several applications.
THe theory is quite applicable in the algebraic theory of rough paths \cite{geng2021introduction}. This has showed to be a very interesting way to represent data because of the special properties of signatures \cite{chevyrev2016primer}. In fact, in a 2013 competition focused on recognizing handwritten Chinese characters, the winner represented the characters as arrays based on a "signature" from rough path theory, then classified them using a convolutional neural network \cite{yin2013icdar}.

\todo[inline]{ Add more application }


We will divide the report into two parts. One part for the fundamental properties of signatures and the second pat for the recent applications of this method.


\todo[inline]{ Main idea is that the signature approach represent a non-parametric way of extraction of characteristics features of data.  }


