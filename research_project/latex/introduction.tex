\section{Introduction}\label{sec:introduction}


Signatures have proven to be fundamental for several applications. The theory is particularly applicable in the algebraic theory of rough paths \cite{geng2021introduction, geng2017reconstruction, fermanian2023new}. This has been shown to be a very interesting way to represent data due to the special properties of time-reversal and the invariance to time reparametrization when representing data via signatures \cite{chevyrev2016primer}. In fact, in a 2013 competition focused on recognizing handwritten Chinese characters, the winning approach represented the characters as arrays based on a "signature" from rough path theory, then classified them using a convolutional neural network \cite{yin2013icdar}. Recently, it has become evident that signatures can be a powerful tool for extracting the geometric shape of any path, even when data is missing.

When the space is discretized and the path is specified by the points it passes through, the resulting representation becomes large and sparse, leading to slow learning. The signature offers a more efficient way to represent a path, helping to mitigate these issues. Importantly, similar smooth curves produce similar signatures. Additionally, the signature is invariant under translations and reparametrizations of the path, making it a robust representation. This, the signature approach represent a non-parametric way of extraction of characteristics features of data.


W fundamental properties e will divide the report into two parts. One part for the  definitions of signatures, and then state the fundamental properties

and the second pat for the recent applications of this method. Spend some time evaluating signatures for

An advantage of this extra theory is that it is generalized and can handle not only Brownian motion but also semimartingales, Markov processes, and Gaussian processes.


