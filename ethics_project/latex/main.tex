

% \documentclass[12pt]{article}


% https://cloudnet2016.ieee-cloudnet.org/authors/final-submission/
\documentclass[10pt,twocolumn]{article}


\usepackage{isaks_template} % Contains all included packages. See isaks_template.sty.
\usepackage{stix}
\usepackage{cancel}  % for cancelling terms

% latex margins
% \linespread{1.5}
\newgeometry{vmargin={15mm}, hmargin={25mm,37mm}}
%
\title{ {\Large \textbf{The societal implications of theoretical mathematics   }} }
% IF ONE AUTHOR
%\address{Norwegian University of Science and Technology \\
%Department of Mathematical Sciences \\
%{\tt isakhammer@gmail.com}}
%

\begin{document}
\author{
Isak Hammer$^{\star\dagger}$  \\
% {\small Supervisor: André Massing$^{\dagger} $  } \\
{\small  October, 2024}\\
{\footnotesize $^\star$isakhammer@gmail.com }
% \vspace{7mm} %5mm vertical space
{\footnotesize $^\dagger$Department of Mathematics, UiB}\\
}

\maketitle
% Comment this out to remove todos
% https://tex.stackexchange.com/questions/4830/how-to-hide-todo-notes-without-deleting-them-manually
% \renewcommand{\todo}[1]{}
\begin{sloppy}
 \textit{ \textbf{Note.} This article is submitted as an examination for the ethics course in   \\}

% \begin{abstract}
% % Title: The Societal Implications of Theoretical Mathematics

% % The Intellectual Stimulation of Teaching Mathematics
% % How does teaching mathematics as a rigorous and structured language enhance intellectual development? Mathematics trains the human mind in logical reasoning and problem-solving by offering well-defined problems and methods for resolution. This intellectual rigor stimulates critical thinking and abstract reasoning, which benefits not only scientific disciplines but also everyday decision-making and analytical skills.

% % Mathematics as an Art Form
% % Can mathematics be appreciated as an art, even though its beauty may seem exclusive to those deeply immersed in the field? Mathematical discovery often involves uncovering unexpected symmetries, patterns, and elegant solutions, akin to artistic creation. While some results might appear esoteric and accessible only to experts, others—though difficult to prove—can be intuitively understood by a broader audience. Is there a balance between the artistic beauty of mathematics and its potential elitism, and how does this perception impact its appreciation by society?

% % The Time-Lag Phenomenon in Mathematical Applications
% % Why do mathematical theories developed centuries ago often find practical applications only much later? Historically, many mathematical concepts—such as non-Euclidean geometry or number theory—were developed without immediate practical use but later became fundamental to advances in fields like physics, engineering, and cryptography. Does society benefit from such delayed applicability, or does this raise questions about the necessity of theoretical work that may not have immediate applications? Should we, as a society, prioritize funding for such long-term research?
% \end{abstract}

\section*{Quick history of applications due to theoretical mathematics}\label{sec:introduction}

Theoretical mathematics has been a essential subject throughout history, with significant implications for society both from art, curiosity and applications.

Since the dawn of civilization, mathematics has been deeply integrated into the human mind, even as the perception of objects might subjectively change. For example, the categorization of colors varies significantly between Western languages and
Russian \cite{maier2018native}, and even the ancient Greeks did not distinguish between blue or green in their color lexicon \cite{durao2022did}. On the other hand, counting is fundamentally equivalent in all languages. Some indigenous languages
possess distinct terms for numbers ranging from $1$ to $5$. For instance, among Eskimos and native Australians count $11$ as 'two hands and a toe', and other gestures for higher numbers \cite{gow2010short}.  However, these numerical systems tend to lack the structural efficiency required for counting larger quantities or powers, which suggests that they may not be optimal for handling more complex arithmetic tasks \cite{beller2008limits}.

An important breakthrough in history from basic counting to more advanced abstraction is when ancient Egyptians invented a number notation and utilized geometry, basic algebraic manipulations and fractions to compute volumes of pyramids and taxes \cite{imhausen2016mathematics}. It is also clear that had some insight of quite complicated geometrically identities. In fact, there is controversy among scientists regarding whether the Egyptians had knowledge of $\pi$ and the golden ratio $\phi$ while constructing the Great Pyramid of Giza, despite lacking the tools to explicitly state their numerical values \cite{robins1985mathematical}.

This was a inspiration and further inspired by the old Greek school philosophy. For the first time, was appreciated as an abstraction and curiously. Thus, foundational logic rules and axioms where introduced into geometry by Euclidian (300 B.C.),
i.e,
\begin{enumerate}
    \item  A straight line segment can be drawn joining any two points.

\item Any straight line segment can be extended indefinitely in a straight line.

\item Given any straight line segment, a circle can be drawn having the segment as radius and one endpoint as center.

\item All right angles are congruent.

\item If two lines are drawn which intersect a third in such a way that the sum of the inner angles on one side is less than two right angles, then the two lines inevitably must intersect each other on that side if extended far enough. This postulate is equivalent to what is known as the parallel postulate.
\end{enumerate}
For reference, see \cite{weissteinEuclidsPostulates}. This facilitated analytical proofs, ensuring consistency and validity in mathematical reasoning. By enabling the generation of new knowledge through logic and proof, it became a significant step
in theoretical mathematics.

However, Gow \cite{gow2010short} notes that Greek numerical methods remained cumbersome for higher arithmetic, arguing that the development of mathematical progress was hindered by inefficient notation.

Around the same era, Chinese mathematicians developed a theory of additions, subtractions, and division based on the use of counting rods, which proved to be highly efficient. This innovation was crucial for the advancement of Chinese civilization. For any arithmetic problem, the rods were applied, and the knowledge was utilized by engineers and administrators for tasks such as calculating land area, transferring money, and distributing goods among people \cite{yong1996development}.

This was later optimized by the Hindu-Arabic mathematicians numeral system, which is now known as the numbers $0-9$ we used today invented in the early 770s, which proved to be essential to develop a consistent theory of arithmetic
operations, including decimal numbers \cite{kunitzsch2003transmission}.
This concept was quickly adopted by professional mathematicians, including Al-Khwarizmi (ca. 780–850) at the 'House of Wisdom' in Baghdad, which is renowned for its development of algebratic theory for solving quadratic equations of the form $$ax^{2}
+ bx + c = 0$$, where $a$, $b$, and $c$ are positive integers and $x$ is a unknown, produce accurate tables for $\sin(x)$  and $\cos(x)$, and lastly big tables describing the movement of the sun, the moon and the five known planets \cite{van2013history}.


This was eventually the basis needed for the enlightenment mathematicians in the 16th Century, such as Newton and Leibniz. In fact, this stated curiosities about infinity, which later gave some obscure relationships between of a specific
infinitely sequence of fractions and trigonometry.
\begin{equation}
    \label{eq:leib}
    \frac{\pi}{4} = 1 - \frac{1}{3} + \frac{1}{5} - \frac{1}{9}- \ldots
\end{equation}
which is the famous so-called Leibniz formula for $\pi $.

While exploring the mysterious behavior of functions and sequences as variables approached infitinity, it opened up the concept of the infinitesimal. This quest ultimately led to the development of the derivative, formalized by the following limit:
\begin{equation}
    \frac{d}{dx} f(x) = \lim_{h \to 0} \frac{f(x+h) - f(x)}{h}.
\end{equation}
In this expression, the derivative captures the instantaneous rate of change of \( f(x) \), as the small increment \( h \) approaches zero, allowing for a precise understanding of how functions behave at every point.
As a scientific curiosity, this exploration unveiled the geometric relationships among acceleration, velocity, and position:
\begin{equation} \begin{split}
\frac{d}{dt} v(t) & = \lim_{h \to 0} \frac{p(t+h) - p(t)}{h}, \\
\frac{d}{dt} a(t) & = \lim_{h \to 0} \frac{v(t+h) - v(t)}{h},
\end{split} \end{equation}
 enabling to relate the position of an object to its acceleration. This was a fundamental tool, as it became the fundamental language of physics, which later was a precursor industrial revolution and more.

Many of the great mathematicians in history pursued their work out of pure curiosity, yet their abstract ideas eventually became critical to the Industrial Revolution. This brief introduction highlights how theoretical mathematics can drive significant advancements, even if its value is not immediately apparent. In the sections that follow, we will explore why mathematics is more than just a tool for practical applications, and how its impact goes far beyond engineering and technology.


\section*{ Theoretical mathematics for training rigorous thinking }%
% \label{sec:mathematics_for_stimulating_intellectual_mind}


Aside for many of the applications in mathematics, it can be hard to argue why theoretical mathematics is necessary for people which does not directly work with statistical data, mathematical models or engineering.
For instance, it not uncommon to see many young
frustrated high school students
which particular demotivating that
mathematics is mandatory for the admission to university for a student program which barely mathematics at all. This also holds for engineering students, which desperately is crawling to understand abstract mathematical definitions, while they will use computer
simulation programs for the rest of their careers. In this section, we try to argue why this mathematical training has much more implications than just for mathematical models, physics and statistics.

A key aspect of the thought process developed in theoretical mathematics is the ability to distill complex problems down to their essential components. This practice trains us to focus on only what is necessary for solving the problem, fostering clarity and precision in our reasoning.

This skill is invaluable, as it helps develop a disciplined approach to problem-solving, which is crucial not only in mathematics but in many other fields. By cultivating such precision early on, we can better prepare students for complex challenges
across a variety of disciplines. It also clarifies that sometimes disciplines are talking about the same thing, while the notation and definitions is different.

To illustrate, here is example of an abstraction of a practical daily life problem.

\begin{enumerate}[label=\arabic*)]
    \item
You are conducting a survey at a local school to find out students' sports preferences. Out of these, 8 students indicate that they prefer swimming, while 18 students prefer hiking.
Additionally, you discover that 3 students enjoy both activities.
    \item Let \( A \) and \( B \) be two sets. The number of elements in \( A \) is \( |A| = 8 \), while the number of elements in \( B \) is \( |B| = 18 \). The number of elements in the intersection of \( A \) and \( B \) is \( |A \cap B| = 3 \).

\end{enumerate}

While the first problem formulation contains a lot of information, the second formulation is mathematically equivalent in principle. By introducing notation, we have assigned groups of students to sets \( A \) and \( B \); however, it is now very
precise and concise, making it easier to discuss what happens with the students in groups \( A \) or \( B \).
This is a simple example of the abstractions that can occur everywhere. However, keep in mind that this example may be a bit inflated, but the point is that some abstraction is often necessary for us to communicate efficiently.


Intuition is sometimes very important to understand the big picture, however, it can sometimes trick you to reach the wrong conclusion. Essentially, all mathematics proofs should be independent of visual representation, but however . \textit{"The point of rigour is not to
destroy all intuition; instead, it should be used to destroy bad intuition while clarifying and elevating good intuition." Terence Tao }\cite{Tao2022}.
Two illustrate, we have here very easy examples \begin{align}
    \frac{78}{17}  &=  \frac{ \cancel{7}8}{ 1 \cancel{7}} = \frac{8}{1} = 8 \\
    \frac{19}{95}  &=  \frac{ 1\cancel{9}}{  \cancel{9} 5} = \frac{1}{5}
\end{align}

What is particularly dangerous about the calculations above is that a method based on completely incorrect intuition still arrives at the correct answer. On these special cases, one might be tricked to believe that the intuitive understanding cause
and effect has higher generality
than in the reality.

To illustrate the importance of rigour; Low carbohydrate diets has a tendency to be correlated to weight loss. While this is likely true, one can with rigour attitude find other possible explanation such that 1) carbohydrates is
likely a significant part of ones diet, 2) Focusing on diet may imply that one will unconsciously  also be motivated train more, which both are contributors to weight loss.

By applying rigour, we do not stop at the surface observation but instead explore underlying mechanisms and potential confounding factors. Rigour ensures that conclusions are not drawn prematurely or based on incomplete reasoning, fostering a deeper
and more accurate understanding of the phenomenon. Here is mathematics a framework where this lesson is tested in almost every problem.


Another useful tool due to mathematics is understanding of nonlinear relationships between variables. As humans we tend to intuitively think linearly, that means if I want to double a pizza dough recipe, we just double the amount of flour, yeast, salt and water. However, it turns out
that doubling yeast has a nonlinear effect and, thus, leads to much faster fermentation causing oversized air pockets and weakened gluten structure.

Not being aware of when systems behave nonlinearly can also, in fact, be life-threatening. Consider you drive on a car with velocity $ v_{0} =  80 km /h $ and you increase the speed to $v_{1} =100 km/h$ to pass a car on the road, that is increasing the
speed only $v_{1}/v_{0} = 125 \%$ . For the untrained mathematician this might
seem fine, since one might think that braking length is proportional to the velocity.

However, from basic physics we have the translational energy function \begin{equation}
    E( v)  = \frac{1}{2} m v^2
\end{equation}
Here $m$ is the constant mass of the car, and $v$ is the velocity.
Due to quadratic growth we get the relative energy growth $$\frac{ E( v_{1} )}{E( v_{0})} = \frac{v_{1}^2}{v_{0}^2 } \approx 156 \%.     $$ Hence, by increasing the speed by only $20km /h$ we added $   56\% $ more energy to the system.
This implies that the work done by the brake length is (which is the energy required to make the car to full stop), in fact, not linear, but quadratic relationship with the velocity.
This can grossly make us underestimate the total brakelength on wet versus dry road conditions, where the brake length is already twice as long due to less friction.

While this is just some examples, this abstract thinking can be applied to tons of real life problems.



\section*{Mathematics as an Art form}%
\label{sec:mathematics_for_stimulating_intellectual_mind}


It is widely known that curiosity is genetically imprinted in our DNA from the very beginning.
In fact, new study shows that even zebrafish voluntarily engage in cognitive stimulation opportunities, indicating their curiosity \cite{franks2023curiosity}.
As human beings, we are a bit more advanced because we can express ourselves in various formats, such as narratives through oral or written language (fairy tales, religion, legends), dance, or drawings. We, as humans, tend to enjoy puzzles and games.
Even my grandmother, with no formal education, enjoys playing Scrabble and Sudoku, among others.
As humans, we enjoy socialising and sharing experiences. We also tend to spend to enjoy good food and watch movies. It is quite clear, humans are fundamentally thriving to stimulate the curiosity.

This might be the reason why art has become a essential part of the human civilisation, even though the applications for it in the society is not entirely obvious.
In this sense, one can argue that mathematics in a framework to express curiosity by
solving problems and discover symmetries.

In contrast to story telling, in the form of movies, games, fashion, novels, music and photography, we see that hard abstract theoretical mathematics can be very hard to appreciate and reachy for the general human being in a society. However,
sometimes it turns out that even though that the foundations is extremely rigour, any person which has some basic knowledge of math can understand it.

Recalling the example of Leibniz's formula \eqref{eq:leib}, it is not hard for the curious to wonder why such a formula involving rational numbers and $\pi$ holds true in the case of an infinitely long sequence of numbers. In theoretical mathematics,
this is just the tip of the iceberg, as it serves as an diverse and uniform language through which one can explain simple results that arise from complex theories, as well as uncover unexpected connections between mathematical fields.

However, to deeply understand it by the ones with enough patient to learn and understand the rules can explain why that sequence must hold.
     But is it egoistic that theoretical mathematicians train for years and years, just to construct some weird symmetry only they and few other well find interesting?

In the famous essay, \textit{A Mathematician's Apology}, British mathematician G. H. Hardy states the following:
\begin{quote}
``... at present I will say only that if a chess problem is, in the crude sense, ‘useless’, then that is equally true of most of the best mathematics; that very little of mathematics is useful practically, and that that little is comparatively dull.

The ‘seriousness’ of a mathematical theorem lies, not in its practical consequences, which are usually negligible, but in the significance of the mathematical ideas which it connects. We may say, roughly, that a mathematical idea is ‘significant’ if it can be connected, in a natural and illuminating way, with a large complex of other mathematical ideas. Thus a serious mathematical theorem, a theorem which connects significant ideas, is likely to lead to important advance in mathematics itself and even in other sciences."
\end{quote}

Some might argue that dedicating time to mathematics is part of a long-standing tradition, where each mathematician throughout history has contributed to the discovery of a vast and elegant abstract world.




    \printbibliography
\end{sloppy}

\end{document}
