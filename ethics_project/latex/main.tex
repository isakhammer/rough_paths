

% \documentclass[12pt]{article}


% https://cloudnet2016.ieee-cloudnet.org/authors/final-submission/
\documentclass[10pt,conference]{IEEEtran}


\usepackage{isaks_template} % Contains all included packages. See isaks_template.sty.
\usepackage{stix}

% latex margins
% \linespread{1.5}
\newgeometry{vmargin={15mm}, hmargin={25mm,37mm}}
%
\title{ {\Large \textbf{The Unreasonable Societal Implications of Theoretical Mathematics   }} }
% IF ONE AUTHOR
%\address{Norwegian University of Science and Technology \\
%Department of Mathematical Sciences \\
%{\tt isakhammer@gmail.com}}
%

\begin{document}
\author{
Isak Hammer$^{\star\dagger}$  \\
% {\small Supervisor: André Massing$^{\dagger} $  } \\
{\small  October, 2024}\\
{\footnotesize $^\star$isakhammer@gmail.com }
% \vspace{7mm} %5mm vertical space
{\footnotesize $^\dagger$Department of Mathematics, UiB}\\
}

\maketitle
% Comment this out to remove todos
% https://tex.stackexchange.com/questions/4830/how-to-hide-todo-notes-without-deleting-them-manually
% \renewcommand{\todo}[1]{}
\begin{sloppy}
 \textit{ \textbf{Note.} This article is submitted as an examination for the ethics course in   \\}

\begin{abstract}
% Title: The Societal Implications of Theoretical Mathematics

% The Intellectual Stimulation of Teaching Mathematics
% How does teaching mathematics as a rigorous and structured language enhance intellectual development? Mathematics trains the human mind in logical reasoning and problem-solving by offering well-defined problems and methods for resolution. This intellectual rigor stimulates critical thinking and abstract reasoning, which benefits not only scientific disciplines but also everyday decision-making and analytical skills.

% Mathematics as an Art Form
% Can mathematics be appreciated as an art, even though its beauty may seem exclusive to those deeply immersed in the field? Mathematical discovery often involves uncovering unexpected symmetries, patterns, and elegant solutions, akin to artistic creation. While some results might appear esoteric and accessible only to experts, others—though difficult to prove—can be intuitively understood by a broader audience. Is there a balance between the artistic beauty of mathematics and its potential elitism, and how does this perception impact its appreciation by society?

% The Time-Lag Phenomenon in Mathematical Applications
% Why do mathematical theories developed centuries ago often find practical applications only much later? Historically, many mathematical concepts—such as non-Euclidean geometry or number theory—were developed without immediate practical use but later became fundamental to advances in fields like physics, engineering, and cryptography. Does society benefit from such delayed applicability, or does this raise questions about the necessity of theoretical work that may not have immediate applications? Should we, as a society, prioritize funding for such long-term research?
\end{abstract}

\section{Introduction}\label{sec:introduction}

Theoretical mathematics is sort of a controversial subject throughout time with big implications on the society. From the start of dawm, where whole num egyptians utilized geometry to construct perfect square pyramides and astrology, it is now clear



\newpage
\section{Mathematics for stimulating intellectual mind}%
\label{sec:mathematics_for_stimulating_intellectual_mind}

\begin{itemize}
    \item For some it may be because we humans tend to like puzzles and games. Even my grandmas enjoy playing Scrabble and sudoko and more. But while this is essentially basic structures and have basic applications, is this evolved from human creativity.
\end{itemize}


\newpage
\section{Mathematics as an Art form}%
\label{sec:mathematics_for_stimulating_intellectual_mind}

Beauty of mathematics, but why does this
\begin{itemize}
    \item The beautiful symmetries that happens in theoretical mathematics.
        \begin{itemize}
            \item Unexpected connections between mathematical fields.
            \item Unexpected simple results, arriving from a complex theory
        \end{itemize}
    \item Is it egoistic that theoretical mathematicians train for years and years, just to construct some weird symmetry only they find interesting.
        \begin{itemize}
            \item Only appreciated by the ones with enough patients to learn the hidden rules
            \item
        \end{itemize}
\end{itemize}

\newpage
\section{Mathematics as an application}%
\label{sec:mathematics_for_stimulating_intellectual_mind}

\begin{itemize}
    \item Time lag before the applications
        \begin{itemize}
            \item The history of how abstract algebra was the only branch where "pure" branch with no applications. But then we found extreme applications in cryptography.
        \end{itemize}
\end{itemize}






    \printbibliography
\end{sloppy}

\end{document}
