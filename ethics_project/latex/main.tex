

% \documentclass[12pt]{article}


% https://cloudnet2016.ieee-cloudnet.org/authors/final-submission/
\documentclass[10pt,conference]{IEEEtran}


\usepackage{isaks_template} % Contains all included packages. See isaks_template.sty.
\usepackage{stix}

% latex margins
% \linespread{1.5}
\newgeometry{vmargin={15mm}, hmargin={25mm,37mm}}
%
\title{ {\Large \textbf{The Unreasonable Societal Implications of Theoretical Mathematics   }} }
% IF ONE AUTHOR
%\address{Norwegian University of Science and Technology \\
%Department of Mathematical Sciences \\
%{\tt isakhammer@gmail.com}}
%

\begin{document}
\author{
Isak Hammer$^{\star\dagger}$  \\
% {\small Supervisor: André Massing$^{\dagger} $  } \\
{\small  October, 2024}\\
{\footnotesize $^\star$isakhammer@gmail.com }
% \vspace{7mm} %5mm vertical space
{\footnotesize $^\dagger$Department of Mathematics, UiB}\\
}

\maketitle
% Comment this out to remove todos
% https://tex.stackexchange.com/questions/4830/how-to-hide-todo-notes-without-deleting-them-manually
% \renewcommand{\todo}[1]{}
\begin{sloppy}
 \textit{ \textbf{Note.} This article is submitted as an examination for the ethics course in   \\}

\begin{abstract}
% Title: The Societal Implications of Theoretical Mathematics

% The Intellectual Stimulation of Teaching Mathematics
% How does teaching mathematics as a rigorous and structured language enhance intellectual development? Mathematics trains the human mind in logical reasoning and problem-solving by offering well-defined problems and methods for resolution. This intellectual rigor stimulates critical thinking and abstract reasoning, which benefits not only scientific disciplines but also everyday decision-making and analytical skills.

% Mathematics as an Art Form
% Can mathematics be appreciated as an art, even though its beauty may seem exclusive to those deeply immersed in the field? Mathematical discovery often involves uncovering unexpected symmetries, patterns, and elegant solutions, akin to artistic creation. While some results might appear esoteric and accessible only to experts, others—though difficult to prove—can be intuitively understood by a broader audience. Is there a balance between the artistic beauty of mathematics and its potential elitism, and how does this perception impact its appreciation by society?

% The Time-Lag Phenomenon in Mathematical Applications
% Why do mathematical theories developed centuries ago often find practical applications only much later? Historically, many mathematical concepts—such as non-Euclidean geometry or number theory—were developed without immediate practical use but later became fundamental to advances in fields like physics, engineering, and cryptography. Does society benefit from such delayed applicability, or does this raise questions about the necessity of theoretical work that may not have immediate applications? Should we, as a society, prioritize funding for such long-term research?
\end{abstract}

\section{Quick history of applications due to theoretical mathematics}\label{sec:introduction}

Theoretical mathematics has been a essential subject throughout history, with significant implications for society both from art, curiosity and applications.

Since the dawn of civilization, mathematics has been deeply integrated into the human mind, even as the perception of objects might subjectively change. For example, the categorization of colors varies significantly between Western languages and
Russian \cite{maier2018native}, and even the ancient Greeks did not distinguish between blue or green in their color lexicon \cite{durao2022did}. On the other hand, counting is fundamentally equivalent in all languages. Some indigenous languages
possess distinct terms for numbers ranging from $1$ to $5$. For instance, among Eskimos and native Australians count $11$ as 'two hands and a toe', and other gestures for higher numbers \cite{gow2010short}.  However, these numerical systems tend to lack the structural efficiency required for counting larger quantities or powers, which suggests that they may not be optimal for handling more complex arithmetic tasks \cite{beller2008limits}.

An important breakthrough in history from basic counting to more advanced abstraction is when ancient Egyptians invented a number notation and utilized geometry, basic algebraic manipulations and fractions to compute volumes of pyramids and taxes \cite{imhausen2016mathematics}. It is also clear that had some insight of quite complicated geometrically identities. In fact, there is controversy among scientists regarding whether the Egyptians had knowledge of $\pi$ and the golden ratio $\phi$ while constructing the Great Pyramid of Giza, despite lacking the tools to explicitly state their numerical values \cite{robins1985mathematical}.

This was a inspiration and further inspired by the old Greek school philosophy. For the first time, was appreciated as an abstraction and curiously. Thus, foundational logic rules and axioms where introduced into geometry by Euclidian (300 B.C.),
i.e,
\begin{enumerate}
    \item  A straight line segment can be drawn joining any two points.

\item Any straight line segment can be extended indefinitely in a straight line.

\item Given any straight line segment, a circle can be drawn having the segment as radius and one endpoint as center.

\item All right angles are congruent.

\item If two lines are drawn which intersect a third in such a way that the sum of the inner angles on one side is less than two right angles, then the two lines inevitably must intersect each other on that side if extended far enough. This postulate is equivalent to what is known as the parallel postulate.
\end{enumerate}
For reference, see \cite{weissteinEuclidsPostulates}. This facilitated analytical proofs, ensuring consistency and validity in mathematical reasoning. By enabling the generation of new knowledge through logic and proof, it became a significant step
in theoretical mathematics.

However, Gow \cite{gow2010short} notes that Greek numerical methods remained cumbersome for higher arithmetic, arguing that the development of mathematical progress was hindered by inefficient notation.

Around the same era, Chinese mathematicians developed a theory of additions, subtractions, and division based on the use of counting rods, which proved to be highly efficient. This innovation was crucial for the advancement of Chinese civilization. For any arithmetic problem, the rods were applied, and the knowledge was utilized by engineers and administrators for tasks such as calculating land area, transferring money, and distributing goods among people \cite{yong1996development}.

This was later optimized by the Hindu-Arabic mathematicians numeral system, which is now known as the numbers $0-9$ we used today invented in the early 770s, which proved to be essential to develop a consistent theory of arithmetic
operations, including decimal numbers \cite{kunitzsch2003transmission}.
This concept was quickly adopted by professional mathematicians, including Al-Khwarizmi (ca. 780–850) at the 'House of Wisdom' in Baghdad, which is renowned for its development of algebratic theory for solving quadratic equations of the form $$ax^{2}
+ bx + c = 0$$, where $a$, $b$, and $c$ are positive integers and $x$ is a unknown, produce accurate tables for $\sin(x)$  and $\cos(x)$, and lastly big tables describing the movement of the sun, the moon and the five known planets \cite{van2013history}.


This was eventually the basis needed for the enlightenment mathematicians in the 16th Century, such as Newton and Leibniz. In fact, this stated curiosities about infinity, which later gave some obscure relationships between of a specific
infinitely sequence of fractions and trigonometry.
\begin{equation}
    \frac{\pi}{4} = 1 - \frac{1}{3} + \frac{1}{5} - \frac{1}{9}- \ldots
\end{equation}
which is the famous so-called Leibniz formula for $\pi $.

While exploring the mysterious behavior of functions and sequences as variables approached infitinity, it opened up the concept of the infinitesimal. This quest ultimately led to the development of the derivative, formalized by the following limit:
\begin{equation}
    \frac{d}{dx} f(x) = \lim_{h \to 0} \frac{f(x+h) - f(x)}{h}.
\end{equation}
In this expression, the derivative captures the instantaneous rate of change of \( f(x) \), as the small increment \( h \) approaches zero, allowing for a precise understanding of how functions behave at every point.
As a scientific curiosity, this exploration unveiled the geometric relationships among acceleration, velocity, and position:
\begin{equation} \begin{split}
\frac{d}{dt} v(t) & = \lim_{h \to 0} \frac{p(t+h) - p(t)}{h}, \\
\frac{d}{dt} a(t) & = \lim_{h \to 0} \frac{v(t+h) - v(t)}{h},
\end{split} \end{equation}
 enabling to relate the position of an object to its acceleration. This was a fundamental tool, as it became the fundamental language of physics, which later was a precursor industrial revolution and more.

Many of the great mathematicians in history pursued their work out of pure curiosity, yet their abstract ideas eventually became critical to the Industrial Revolution. This brief introduction highlights how theoretical mathematics can drive significant advancements, even if its value is not immediately apparent. In the sections that follow, we will explore why mathematics is more than just a tool for practical applications, and how its impact goes far beyond engineering and technology.


\newpage
\section{Mathematics for stimulating intellectual mind}%
\label{sec:mathematics_for_stimulating_intellectual_mind}

\begin{itemize}
    \item For some it may be because we humans tend to like puzzles and games. Even my grandmas enjoy playing Scrabble and sudoko and more. But while this is essentially basic structures and have basic applications, is this evolved from human creativity.
\end{itemize}


\newpage
\section{Mathematics as an Art form}%
\label{sec:mathematics_for_stimulating_intellectual_mind}

Beauty of mathematics, but why does this
\begin{itemize}
    \item The beautiful symmetries that happens in theoretical mathematics.
        \begin{itemize}
            \item Unexpected connections between mathematical fields.
            \item Unexpected simple results, arriving from a complex theory
        \end{itemize}
    \item Is it egoistic that theoretical mathematicians train for years and years, just to construct some weird symmetry only they find interesting.
        \begin{itemize}
            \item Only appreciated by the ones with enough patients to learn the hidden rules
            \item
        \end{itemize}
\end{itemize}

\newpage






    \printbibliography
\end{sloppy}

\end{document}
